\documentclass[a4paper, 12pt]{article}

\usepackage[margin=1in]{geometry}
\usepackage{graphicx}
\usepackage{subcaption}

\graphicspath{{Images/}}

\begin{document}
    \begin{titlepage}
        \begin{flushleft}
            \includegraphics[width=0.3\textwidth]{UNM logo.png} 
        \end{flushleft}
        \vspace{3cm}
        
        \begin{center}
            {\Huge \bfseries Log Book} 
            \par \vspace{\baselineskip}
            {\Large Project Weeks} 
            
            \vfill
            
            {Mohamed Nabeeh Ibrahim} \par
            {20786034}
        \end{center}
    \end{titlepage}

    \section{Week One}
    \subsection{Day One}
    Built the H-Bridge motor controller by referring to the circuit diagram given in the lab sheet. The components used were breadboard, DC motor MM10,
    4 slide switches and jumper wires.

    \begin{figure}[h]
        \centering
        \begin{subfigure}[h]{0.3\textwidth}
            \includegraphics[width=\textwidth]{H-Bridge motor driver on breadboard.jpg}
            \caption{H-Bridge motor driver on breadboard}
            \label{fig:H-Bridge breadboard}
        \end{subfigure}
        \hspace{2cm}
        \begin{subfigure}[h]{0.3\textwidth}
            \includegraphics[width=\textwidth]{H-Bridge in tinkercad.png}
            \caption{H-Bridge in tinkercad}
            \label{fig:H-Bridge tinkercad}
        \end{subfigure}
        \caption{}
    \end{figure}

    Before testing the circuit using the (\ref{fig:H-Bridge breadboard}) power supply, tested the design in tinkercad by modelling the circuit and running a simulation.
    The simulation was a success so connected the terminals of the power supply to the breadboard and set the power supply to output 3v and 0.1A,
    then turned S1 and S3 on. The motor was spinning as expected so turned off S1 and S3, then turned on S2 and S4. The motor started spinning
    in the opposite direction hence the first task was completed.

    Afterwards started to work on the robot car. Attached all four motors to the body and attached all of the wheels to the motors.

    \begin{figure}[h]
        \centering
        \includegraphics[width=0.3\textwidth]{Motor assembly.jpg}
        \caption{Motor assembly}
        \label{fig:Motor assembly}
    \end{figure}

    \subsection{Day Two}
    Connected the wires of the motors to the necessary output terminals of the L298N motor driver. Then connected the battery positive terminal
    to the 12v terminal of the motor driver and the battery negative terminal to the ground terminal of the motor driver. Afterwards
    made connections between the ground and 5v of the motor driver and the arduino. After that connected the enA and enB terminals on the motor driver
    to the A1 and A2 terminals on the arduino. Then connected in1, in2, in3 and in4 on the motor driver to digital pins 2, 3, 4 and 5 on the arduino respectively.

    \begin{figure}[h]
        \centering
        \includegraphics[width=0.3\textwidth]{Motor driver connections.jpg}
        \caption{Motor driver connections}
        \label{fig:Driver connections}
    \end{figure}

    Wrote the code to display the elapsed time on the LCD display and to drive the car forwards and uploaded it to the arduino.

    When testing the car, only the motors on one side were spinning. Learned that since the LCD display is mounted onto the arduino, 
    not all of the digital pins are available. Hence remade some of the connections. Now in1, in2, in3 and in4 on the motor driver were connected to
    digital pins 13, 12, 11 and 4 on the arduino. Modified the code as necessary and uploaded it to the arduino. The car was now working as intended.
\end{document}